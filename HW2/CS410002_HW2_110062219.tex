\documentclass[12pt, a4paper]{article}

\title{\textsc{Computer Architecture} Homework 2}
\author{110062219}
\date{\today}

\usepackage{amsmath}
\usepackage{amssymb}
\usepackage{caption}
\usepackage{subcaption}
\usepackage{tikz}
\usepackage{pgfplots}
\usepackage{listings}
\usepackage{float}
\usepackage{tabularx}
%\usepackage{geometry}[margin=2cm]

\lstset{
  breaklines=true,
  basicstyle=\ttfamily,
}

\definecolor{nthu}{HTML}{7F1084}

% \renewcommand{\ttdefault}{pcr}

\begin{document}

\maketitle

\tableofcontents

\section{Andes C GCD project}

\subsection{Find the memory addresses}

The address of the global variable \texttt{result} is shifted 4968 from the global register \texttt{gp}, which equals \texttt{0x11538}.

The address of the procedure \texttt{gcd} is \texttt{0x104a8}.

\subsection{\texttt{remw} instruction}

The \texttt{remw} instruction performs the \textbf{modulus} operation on \textbf{words}, i.e., 32-bit integers. Specifically, we do signed division on the lower 32 bits of the source registers and store their remainder with sign extension to the destination register.

\subsection{Stack}

\subsubsection{Stack frame}

\begin{verbatim}
  c.addi16sp sp,-48
  c.sdsp ra,40(sp)
  c.sdsp s0,32(sp)
  c.addi4spn s0,sp,48
  c.mv a5,a0
  c.mv a4,a1
  sw a5,-36(s0)
  c.mv a5,a4
  sw a5,-40(s0)
  lw a4,-36(s0)
  lw a5,-40(s0)
  remw a5,a4,a5
  sw a5,-20(s0)
\end{verbatim}

When entering the \texttt{gcd()} function, we would allocate a stack frame of size 48 and push the previous \textit{return address}, \textit{frame pointer} to the top of it.

Moreover, we would save the local variables $a$, $b$ and the register that stores $a\bmod b$ into the frame as well.

\begin{table}[H]
\caption{Stack block}
\centering
\begin{tabular}{cccc}
\hline
Addresses & Contents & Offsets & Notes \\
\hline\hline
High & & 0 & \textit{Frame pointer} \\
 & Saved \textit{return address} & -8 & \\
 & Saved \textit{frame pointer} & -16 & \\
 & Saved $a\bmod b$ & -20 & \\
 & & -32 & \\
 & Saved $a$ & -36 & \\
 & Saved $b$ & -40 & \\
Low & & -48 & \textit{Stack pointer} \\
\hline
\end{tabular}
\label{stack frame}
\end{table}

\subsubsection{Lowest address of stack pointer}

The depth of the recursive calls is 6. Hence the lowest address \textit{stack pointer} would reach should be \texttt{0x2FFFED0}.

\subsection{Assembly code corresponding to the tail recursion}

\begin{verbatim}
  lw a4,-20(s0)
  lw a5,-40(s0)
  c.mv a1,a4
  c.mv a0,a5
  jal ra,0x104a8 <gcd>
  add a5,zero,a0
\end{verbatim}

First we load $b,a\bmod b$ from memory we stored previously. Then we copy them to \texttt{a0}, \texttt{a1} and do the recursive call.

\subsection{Assembly code corresponding to the tail recursion with \texttt{-Og}}

\begin{verbatim}
  c.mv a0,a5
  jal ra,0x104a8 <gcd>
\end{verbatim}

We copy $b$ to \texttt{a0}. Since the remainder has been stored in \texttt{a1}, we could thus do the recursive call. With \texttt{-Og}, we make better use of registers and reduce the access to memory.

\section{RISC-V codes}

\subsection{Hexadecimal representation}

The offset is -6, \texttt{0b1111\_1110\_1000} in 12-bit 2's complement, so we have the table:

\begin{table}[htp]
\caption{Binary representation}
\centering
\begin{tabular}{cccccccc}
\hline
$imm_{12}$ & $imm_{10:5}$ & $rs_2$ & $rs_1$ & $funct_3$ & $imm_{4:1}$ & $imm_{11}$ & $opcode$ \\
\hline\hline
1 & 111111 & 00000 & 01010 & 000 & 0100 & 1 & 1100011 \\
\hline
\end{tabular}
\label{beq}
\end{table}
, which is equivalent to \texttt{0xfe0504e3}.

\subsection{Jump-and-link instruction}

We would jump to the second next instruction \texttt{00000000101000011011110000100011}, which indicates \texttt{sd a0,24(gp)}.

\subsection{Long jump}

The address of \texttt{beq} is \texttt{0x80000038}.

\begin{table}[htp]
\caption{Long jump instructions}
\centering
\ttfamily
\begin{tabular}{cl}
\hline
Addresses & Instructions \\
\hline\hline
0x20000000 & lui t0,0x80000 \\
0x20000004 & jalr zero,0x038(t0) \\
\hline
\end{tabular}
\label{long jump}
\end{table}

After \texttt{jalr}, the \texttt{PC} would become \texttt{0x80000038}.

\section{Endianness}

Since an ASCII character is a single byte, it makes no difference whether it's little or big endian.

\begin{table}[H]
\caption{Memory addresses \& ASCII data}
\centering
\ttfamily
\begin{tabular}{cc}
\hline
Addresses & Data \\
\hline\hline
0x00000000 & 0x52 \\
0x00000001 & 0x49 \\
0x00000002 & 0x53 \\
0x00000003 & 0x43 \\
0x00000004 & 0x2D \\
0x00000005 & 0x56 \\
0x00000006 & 0x32 \\
0x00000007 & 0x76 \\
\hline
\end{tabular}
\label{endian}
\end{table}

\section{C to RISC-V}

\begin{verbatim}
slli t0,x5,4 # (i*2) * 8 = i * 16
add t0,x6,t0 # t0 = &A[i * 2]
ld t0,0(t0)  # t0 = A[i * 2]

slli t0,t0,3
add t0,x7,t0 # t0 = &B[A[i * 2]]
ld t0,0(t0)  # t0 = B[A[i * 2]]

addi x10,t0,-16
\end{verbatim}

\section{RISC-V to C}

I name the pointer \texttt{x30}, variable \texttt{x31} to be \texttt{tmp0}, \texttt{tmp1} respectively.

\begin{verbatim}
tmp0 = D;                    // addi x30,x14,0
for (i = 0; i < m; i++)      // addi x11,x0,0; bge x11,x3,ENDI; addi x11,x11,1
{                            // LOOPI:
    tmp1 = *tmp0;            // lw x31,0(x30)
    for (j = 0; j < n; j++)  // addi x12,x0,0; bge x12,x4,ENDJ; addi x12,x12,1
    {                        // LOOPJ:
        tmp1 += (i + j) * 7; // add x27,x11,x12; slli x28,x27,3; sub x28,x28,x27;
                             // add x31,x31,x28
    }                        // jal x0,LOOPJ; ENDJ:
    *tmp0++ = tmp1;          // sw x31,0(x30); addi x30,x30,4
}                            // jal x0,LOOPI; ENDI:
\end{verbatim}

We could further simplify:

\begin{verbatim}
for (i = 0; i < m; i++)
    for (j = 0; j < n; j++)
        D[i] += (i + j) * 7;
\end{verbatim}

\section{C to RISC-V again}

\begin{verbatim}
func:
  addi sp,sp,-32
  sd ra,24(sp)
  sd s0,16(sp)
  sd s1,8(sp)
  sd s2,0(sp)
  addi s0,a0,0    # mv s0,a0
  addi a5,zero,2  # li a5,2
  bltu a5,a0,L2   # bgtu a0,a5,.L2
L1:
  ld ra,24(sp)
  ld s0,16(sp)
  ld s1,8(sp)
  ld s2,0(sp)
  addi sp,sp,32
  jalr zero,0(ra) # jr ra
L2:
  addiw a0,a0,-1
  jal ra,func     # call func
  mulw s2,s0,s0
  add s2,s2,a0
  addiw a0,s0,-2
  jal ra,func     # call func
  slli s1,a0,3
  add s1,s1,s2
  addiw a0,s0,-3
  jal ra,func     # call func
  add a0,s1,a0
  jal zero,L1     # j .L1
\end{verbatim}

\section{True or false}

\subsection{Big-endian}

False. We would get \texttt{0xB3}.

\subsection{Temporary registers}

False. Caller should save \texttt{t0}, \texttt{t1} if necessary.

\subsection{Load upper immediate}

False. \texttt{x6} would become \texttt{0xFFFFAAAA}.

\subsection{Memory access}

True. 2 \texttt{ld} and a \texttt{sd} add up accessing the memory thrice.

\end{document}
